\documentclass[a4paper,12pt]{article}

\usepackage[
	%iwona,									% použije se font Iwona jako hlavní font (okrasné)
	%final,									% pro finální sazbu (nepoužijí se barevné odkazy atp.), knižní sazba
	blacklogo,								% černé UPOL logo místo bílého
	%joinlists,								% zobrazí seznamy (zkratek, obrázků...) pod sebou
	abbreviations,							% povolí podporu zkratek (seznam zkratek, deklarace zkratek)
	figures,								% povolí seznam obrázků
	tables,									% povolí seznam tabulek
	listings,								% povolí seznam zdr. kódů
	language=czech,							% jazyk práce
	bibfile=bibliography.bib,				% soubor s citacemi (BibLaTeX)
	bibencoding=utf8,						% kódování souboru s citacemi
	bibstyle=numeric,						% styl citování (dále například authoryear nebo philosophy-modern)
	encoding=utf8,							% kódování tohoto a vložených souborů
	]{upbase}

\usepackage[
	master									% píšeme magisterskou (diplomovou) práci
	]{updiplom}

\uptitle{Dokonalá práce}						% název práce
\upsubtitle{Opravdu dokonalá}				% podtitulek práce - použijte \upsubtitle{\null} pokud nemáte v práci podnadpisek
\upyear{\the\year}
\upauthor{Martin Rotter}
\upclass{APLINF, III. ročník}
\upleader{Mgr. Petr Velký}
\upanot{Tento dokument je fajn.}			% anotace práce - obvykle jeden odstavec textu
\upthanks{Děkuji všem. Hlavně sobě.}						% poděkování

\usepackage{multirow}
\usepackage{ltxdockit}

\begin{document}

% tiskne titulní stranu
\upmaketitle

% tiskne dvě strany s anotací a poděkováním
\upthanksanot

% tiskne obsah a seznamy obrázku, tabulek a případně zdr. kódů, více v makrech \uptableofcontents a \upprintlists
\uptocandlists

\section{Úvod}
Tohle je úvod.

\section{Další sekce}
Tohle je další sekce

\begin{uplemma}[Démonické lemma]
Naše nové lemma.
\end{uplemma}

\begin{uptheorem}[Název definice]
Naše nová definice.
\end{uptheorem}

\begin{upproof}[Název důkazu]
Náš nový důkaz.
\end{upproof}

\begin{upquote}[Pumpovací věta]
Náš nový důkaz.
\end{upquote}

% tiskne český závěr práce
\begin{upconclusions}[czech]
Závěr práce v \enquote{českém} jazyce.
\end{upconclusions}

% tiskne anglický závěr práce
\begin{upconclusions}[english]
Thesis conclusions written in \enquote{english}.
\end{upconclusions}

% tiskne seznam zkratek
% seznam zkratek se netiskne před práci ale až nakonec
\upprintabbrevlist

% tiskne seznam teorémů
\upprinttheoremlist

% tiskne bibliografii
% bibliografie používá BibLaTeX
\upprintbibliography

% tiskne přílohy
\upappendix
\section{První příloha}
Text první přílohy

\section{Druhá příloha}
Text druhé přílohy

% tiskne rejstřík
\upprintindex

\end{document}