\documentclass[a4paper,12pt]{article}

\usepackage[
	figures,
	tables,
	listings,
	language=czech,
	bibfile=bibliography.bib,
	bibencoding=utf8,
	encoding=utf8,
	]{updiplomex}

\uptitle{Název práce}						% název práce
\upsubtitle{Podtitulek práce}				% podtitulek práce - použijte \upsubtitle{\null} pokud nemáte v práci podnadpisek
\upyear{\the\year}
\upauthor{Martin Rotter}
\upanot{Tento dokument je fajn.}			% anotace práce - obvykle jeden odstavec textu
\upthanks{Děkuji všem.}						% poděkování

\usepackage{multirow}
\usepackage{lipsum}
\usepackage{longtable}

\begin{document}

% tiskne titulní stranu
\upmaketitle

% tiskne dvě strany s anotací a poděkováním
\upthanksanot

% tiskne obsah a seznamy obrázku, tabulek a případně zdr. kódů, více v makrech \uptableofcontents a \upprintlists
\uptocandlists

\section{Styl updiplomex}
Styl \keyword{updiplomex} je novou verzí původních \LaTeX stylů pro psaní diplomových prací a nabízí některá vylepšení a další finesy.

\section{Podpora ovladačů a prostředí}
V současné době byla zahozena podpora \keyword{cslatex} a \keyword{pdfcslatex}. Ke kompilaci dokumentů je tedy používán výlučně \keyword{latex} a \keyword{pdflatex}. Toto použití má hned několik důvodů:
\begin{itemize}
\item není důvod obecně nepoužívat PDF jako přímý výstup (konverze PDF $\rightarrow$ PS jsou bezbolestné)
\item nepoužití cslatex-u přináší daší bonusy, jako například multijazykové možnosti s balíkem \keyword{babel}
\item pdflatex je moderní
\end{itemize}

\section{Možnosti balíku}
Balíku updiplomex je možno volat s několika vstupními argumenty, které mohou zásadním způsobem ovlivnit vzhled dokumentu. Každý argument je popsán v následujícím výčtu a je uveden ve formátu \verb|název - výchozí hodnota - popis|.

\begin{itemize}
\item[final] -- false -- Je-li true, pak se hypertextové odkazy nebudou obarvovat, což se hodí pro knižní tisk. 
\item[blacklogo] -- false -- Je-li true, tak bude titulní strana obsahovat černé logo místo bílého. 
\item[master] -- false -- Je-li true, tak bude vypsán na titulní straně text \enquote{Diplomová práce}. 
\item[joinlists] -- false -- Je-li true, tak budou seznamy a obsah sdruženy za sebe bez vytváření nových stran. 
\item[figures] -- false -- Je-li true, tak budou seznamy obsahovat seznam obrázků. 
\item[tables] -- false -- Je-li true, tak budou seznamy obsahovat seznam tabulek.  
\item[listings] -- false -- Je-li true, tak budou seznamy obsahovat seznam zdrojových kódů.  
\item[language] -- czech -- Nastavuje jazyk používaný balíkem babel. 
\item[bibfile] -- bibliography.bib -- Nastavuje soubor s bibliografií pro Bib\LaTeX 
\item[bibencoding] -- utf8 -- Nastavuje kódování ono souboru z předchozího řádku. 
\item[bibcitestyle] -- iso-authoryear -- Nastavuje formát citace, další možností je iso-numeric 
\item[titleanot] -- Anotace -- Nastavuje nadpis pro anotaci. 
\item[titlelistings] -- Seznam zdrojových kódů -- Nastavuje nadpis pro seznam zdr. kódů. 
\item[namelistings] -- Zdrojový kód -- Nastavuje nadpisek pro každý zdr. kód, tohle využívá například \verb|\caption| a seznam zdr. kódů. 
\item[titleconclusioncz] -- Závěr -- Nastavuje nadpis českého závěru práce. 
\item[titleconclusionen] -- Conclusions -- Nastavuje nadpis anglického závěru práce. 
\item[titlebookmarktitlepage] -- Titulní strana -- Nastavuje nadpis záložky titulní strany 
\item[encoding] -- utf8 -- nastavuje kódování vstupních tex souborů, tedy i tohoto souboru. 
\end{itemize}

\section{Překlad ukázkového dokumentu}
Překlad dokumentů se provádí před pdflatex\index{pdflatex}, navíc je třeba použít makeindex a biber pro překlad rejstříku a bibliografie. Vše je obsaženo v souboru Makefile.

\begin{upcode}{Překlad dokumentů.}{code:compilation}{make}
all: start guide finish

start:
	echo "Makefile has started its work."

guide:	updiplomex.tex
	pdflatex updiplomex
	pdflatex updiplomex
	makeindex updiplomex.idx -s index.ist
	biber updiplomex
	pdflatex updiplomex
	pdflatex updiplomex

finish:
	echo "Makefile has finished its work."

clean:
	rm -f *.lo* *.aux *.ind *.idx *.ilg *.toc
\end{upcode}

\section{Další doplňky}
\subsection{Užitečná makra}
Balík obsahuje makra pro sázení názvů některých programovacích jazyků, prozatím tu máme makra \verb|\csharp| a \verb|\cpp|, která vykrelsují symboly \csharp a \cpp.

Rovněž výčty obsahují nový úvodní symbol, původní tečka byla nahrazena toutéž (avšak šedou) tečkou.
\begin{itemize}
\item šedá tečka
\end{itemize}

Navíc je zde makro \verb|\keyword|, které vypíše zadaný argument \keyword{šedou barvou v neproporcionálním fontu}.

\section{Bibliografie}
Bibliografie používá balík Bib\LaTeX, který je na správu bibliografie asi v současnosti nejlepší, navíc je implementován citační formát dle ISO690, byť tady jsou zatím viditelné problémy. Nicméně vše by mělo správně fungovat.

\section{Zdrojové kódy}
Je použit balík listings. Pomocí něj lze zobrazovat pěkné zdrojové kódy. Ukázku vidíte například niže. Zatím je nastavena barevná syntaxe pro jazyk \cpp, ale podporována je spousta dalších. 

\begin{upcode}{Ukázkový \cpp kód}{code:cpp}{[GNU]C++}
int main(int argc, char *argv[]) {
	return 0;
}
\end{upcode}

% tiskne český závěr práce
\begin{upconclusions}[czech]
Závěr práce v českém jazyce.
\end{upconclusions}

% tiskne anglický závěr práce
\begin{upconclusions}[english]
Thesis conclucions in english.
\end{upconclusions}

% tiskne bibliografii
% bibliografie používá BibLaTeX
\upprintbibliography

% tiskne přílohy
\upappendix
\section{První příloha}
Text první přílohy

\section{Druhá příloha}
Text druhé přílohy

% tiskne rejstřík
\upprintindex

\end{document}