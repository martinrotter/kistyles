\documentclass[a4paper,12pt]{article}

\usepackage[
	%iwona,									% použije se font Iwona jako hlavní font (okrasné)
	%final,									% pro finální sazbu (nepoužijí se barevné odkazy atp.), knižní sazba
	blacklogo,								% černé UPOL logo místo bílého
	%joinlists,								% zobrazí seznamy (zkratek, obrázků...) pod sebou
	abbreviations,							% povolí podporu zkratek (seznam zkratek, deklarace zkratek)
	figures,								% povolí seznam obrázků
	tables,									% povolí seznam tabulek
	listings,								% povolí seznam zdr. kódů
	language=czech,							% jazyk práce
	bibfile=bibliography.bib,				% soubor s citacemi (BibLaTeX)
	bibencoding=utf8,						% kódování souboru s citacemi
	bibstyle=numeric,						% styl citování (dále například authoryear nebo philosophy-modern)
	encoding=utf8,							% kódování tohoto a vložených souborů
	]{upbase}

\usepackage[
	custompicture=\null						% název souboru pro uživatelský obrázek, který se zobrazí vedle UPOL loga
	]{upsimple}								% přepínače tohoto stylu jsou dále popsány v jeho dokumentaci

\uptitle{Název práce}						% název práce
\upsubtitle{Podtitulek práce}				% podtitulek práce - použijte \upsubtitle{\null} pokud nemáte v práci podnadpisek
\upyear{\today}								% datum publikace práce
\upauthor{
	Martin Rotter \\
	Pepa Novák}								% autoři práce
\upanot{Tento dokument je fajn.}			% anotace (abstrakt) práce - obvykle jeden odstavec textu
\uppapertype{SEMESTR\'ALN\'I PR\'ACE}		% typ práce (PROJEKTOV\'A DOKUMENTACE atp.)

\begin{document}

% vytiskne titulní stranu
\upmaketitle

% vytiskne anotaci (abstrakt)
\upmakeanot

% tiskne obsah a seznamy obrázku, tabulek a případně zdr. kódů, více v makrech \uptableofcontents a \upprintlists
% tiskne pouze povolené seznamy
\uptocandlists

\section{Úvod}
Toto je dokumentace stylu \keyword{upsimple}.

\upabbrevdeclare{UPOL}{UPOL}{Univerzita Palackého v Olomouci}

Pokud například máme definovanou zkratku UPOL, tak se na ní můžeme odkázat několikrát, třeba dvakrát, za sebou. Tedy \upabbrevref{UPOL} a ještě jednou \upabbrevref{UPOL},

\index{hm}.

% vytiskne nepatrnou mezeru do obsahu dokumentu (používá se pro vizuální oddělení) dodatečných částí práce
% od hlavní části práce)
\upendoftreatise

% tiskne přílohy
\upappendix
\section{První příloha}
Text první přílohy

\section{Druhá příloha}
Text druhé přílohy

\upendoftreatise

% pokud některou z následujích featur nepotřebujete tak ji jednoduše zakomentujte a případně zakažte danou funkcionalitu
% viz parametr abbreviatons

% vytiskne seznam zkratek
\upprintabbrevlist

% tiskne seznam teorémů
\upprinttheoremlist

% vytiskne bibliografii
\upprintbibliography

% vytiskne rejstřík
\upprintindex

\end{document}